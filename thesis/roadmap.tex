\newpage

\chapter{Road Map}
\fixme{Remove this from the final version}

\begin{itemize}
\color{green}
\item Create the KD-tree from points on the GPU \checkmark
\item Visualize the KD-tree (as part of debugging), draw frame around
  splitting plane in color corrosponding to splitting axis. (\checkmark)
\item Extend KD-tree to work on triangles \checkmark
\item raytrace the triangle KD-tree \checkmark
\item Add a balanced tree lower node creator \checkmark
\item Reduce more than 1024 segments \checkmark
\item Add empty space splitting
  \begin{itemize}
    \item Extend CreateUpperChildren with the possibility to use indices \checkmark
    \item Copy aabb to lower nodes \checkmark
    \item In reduce, compare the existing aabb with the reduced. \checkmark
    \item Move the nodes by the amount of emptySplitNodes, create the
      empty split nodes in front of the nodes and link them with the
      previous nodes parents.\\ Remember to update segments (used when
      splitting triangles), node size and activeIndex. \checkmark
    \item \color{red}Make dynamic empty space threshold. Higher
      probability of splits at the top, than bottom. Let it adjust
      relative to totalPrims / nodePrims.
    \item PROFIT
  \end{itemize}
\item Optimize GeometryList
  \begin{itemize}
  \item Removed OpenGL and use device arrays instead. Oh such clean
    geometry collection code. \checkmark
  \end{itemize}
\item Partition the screen into 8x4 ray blocks to improve ray
  traversel coalescenece. \checkmark
\item Use triangle-box intersection to remove ``impossible to hit''
  triangles from the lower nodes. Aabb might need to be propagated
  downwards in order to do this.
\item Create an optimized aabb intersection algorithm that
optimizes with respect to a triangle and 2 boxes. Doesn't work due to
the compiler dropping the ball when switches contain huge cases..
\item Use woop triangle intersection instead while raytracing (Make an
  enum that can switch). Place the offset in the fourth component of
  each vector, that way I only need to load 1 float4 pr tHit
  component. Less register usage! Yielded 20\% increase on dragon2 in
  cornell box.
\item Make colors float instead of char. Removes 4 divisions when
  coloring. (And apparently adds a data fetching overhead, not worth
  it)
\item Add a bounding box check before raytracing leafs? Provides an
  early out option for long rays. Nice speedup. Won't work aswell with
  small leaf nodes? Inspired by empty space splitting.
\item Instead of moving nodes when doing an empty split, place them
  empty nodes after index + range and remember to skip them when
  creating children. (node size can be used for this, and the
  outermost while loop works unaltered).
\item Handle axis aligned rays. y = copysignf(x, y);

\color{red}
\item Change BalancedCreator to use the same termination criteria as SAH, but without the area. (So it's basicly H :)
\item Fix KernelConf1D when registers are used (something about 512bit
  allocation, unless 2.0 architeture)
\end{itemize}

If time allows

\begin{itemize}
\item Placing splitting planes dependent on the propagated aabb yields
  a much better result. Bug or explain? 
\item Instead of resizing aabbs, perhaps just use the original size
  and crop it with the propagated aabb. (Only if propagateAabbs is
  true, so template argument ftw) Then test if splitting and creating
  new aabbs even adds any speed or if indexing is faster. Lower
  Creators need to be extended to clip the primitives bounding boxes
  to the nodes aabb.
\item Add triangle split. Triangle split can possibly gain
  some performance, by always doing the ``2 triangles'' side first.
\item Store empty split nodes in shared mem and then dump it all
  coallesced after they have been created.
\item Speed up node child creation by using shared mem as
  cache. Enough threads to fill the GPU could be started and then each
  work on their continous segment of the nodes
\item Speedup shadow rays and ambient occlusion. Mostly from simpler
  intersection testing and less register usage.\\ -- OR -- \\ Perhaps
  let the primary raytracer to it's job (coloring and refraction),
  then dump the individual colors and primitive indices into arrays
  (some MAX like 5?), let a different raytracer pickup these values
  and start tracing shadows. Combine everything in a last kernel for
  PROFIT.
\item Add Persistent Theads to SAH. Extend it with preemtively
  replacing dead threads every 4 or so cycle.
\item Add persistent threads to the ray tracer
\item Add an oscilating surface with nice refraction!!
\item Look into SAH approximations
\item At lower tree levels where SAH isn't computed parallel, perhaps
  and early out ``good enough'' value/ratio can be given, as done in
  BSP. Might only increase instructions, branching and still wait for
  the slowest thread. (Try and watch it fail) Perhaps without an if
  statement but by arithmetic instead?
\item Perhaps empty space splitting can be started in it's own stream?
  At least the actual empty space splitting should be able to. Could
  help out at the early tree creation when the GPU is
  underutilized. Suggest or actually try? Would make it an even
  cheaper optimization.
\item Optimize segmented reduce. (low low priority)
\end{itemize}



\subsection*{Trier}

Hvordan splittes geometry?

Hvordan håndteres nye rays? Skabe dem i et gigantisk array og
compacte?

\subsection*{Sangild}

Hvordan refereres lånte figure?

Hedder det self reflection eller self reflecting? (Allan)

Referer til hvor koden kan findes?
