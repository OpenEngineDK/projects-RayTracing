%% Describe and reference previous work that is relevant to your work.

%% The previous work section is mostly descriptive.

%% Address the weaknesses of the previous methods.

%% You should not do a full comparison between your method and previous
%% work here. Leave that for the Results section.

%% You can however distinguish yourself from previous work by saying
%% something like ”In contrast to method X, my method...”, or ”The main
%% difference between my work and X is...”.

\chapter{Previous Work}

\chapterquote{If you want to make an apple pie from scratch, you must
  first create the universe.}{Carl Sagan}

% Early ray tracing

Arthur Appel is credited as the being the first to describe the basic
idea of ray casting\citebook{Appel:1968}, using it to solve the hidden
surface problem and to compute shadows in opaque polygonal scenes to
enhance the perseption of depth. Whitted extended the idea of ray
casting into the general recursive ray tracing algorithm still used
today\citebook{Whitted:1979}. If a ray would hit a surface, it could
generate any number of new rays depending on the surface's material
proporties, reflection, refraction or shadow.


% Data structures for acceleration

Since then a lot of time and effort has gone into improving the
performance of ray tracing and several data structures have been
proposed with this in mind. In 1976 Clark was the first to suggest
using \textit{bounding volumes} to perform geometry culling. 

% Previously constructing the KD-tree was most effective on the CPU,
% optimized and fast for multi core CPU's
\citebook{1230129}

% GPU solutions had to use other datastructures such as grids before
% GPGPU programming.
\citebook{844181}

% or create hybrid GPU/CPU solutions
\citebook{1572783}


% KD-tree work


% Recent research has made it possible to construct KD-trees
% efficiently on GPGPU's
\citebook{1409079}

% Surface Area Heuristic
