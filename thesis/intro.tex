\chapter{Introduction}

% Why global illumination vs local illumination used in most rasterizers

% Before photon mapping it was raytracing and radiosity

% introduction of photon mapping in 1996
\citealp{275461}

% Photon mapping is 'rendering' independent. Works both with ray
% tracing and rasterization.


% importance of efficient datastructures, several datastructures have
% been used for raytracing (BVH, kd-trees, octrees, bla bla...) Some
% are good for bla, others blank


% Needs to be able to quickly create the data structure aswell for
% photon mapping with dynamic geometry/lighting.

% Needs to exploit the massively parallel architecture of GPGPU's
% which require oh so many thread

% can be done by parallising evaluating the approximate cost of each
% node

\textit{Surface Area Heuristic}, $SAH$, 

\textit{Voxel Volume Heuristic}, $VVH$,

% and creation of nodes at lower tree levels
