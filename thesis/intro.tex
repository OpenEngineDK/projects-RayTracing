\chapter{Introduction}

\chapterquote{Focus is a matter of deciding what things you're not
  going to do.}{John Carmack}

% Why global illumination vs local illumination used in most rasterizers

% Why use the GPU? Offloading. The CPU can do a lot of other things
% while the GPU crunch away at those KD trees. Om nom nom nom nom....

% Why even use datastructures on the GPU. They work wonders on the CPU
% but for the GPU they might hinder performance (cite KNN paper and
% dataparallism example with 31 threads waiting)

% In this thesis a brute force raytracer will first bebuild and
% compared to a basic raytracer traversing a KD-tree, to blank
% emphasize the importance of datastructures, even on the GPU.

% importance of efficient datastructures, several datastructures have
% been used for raytracing (BVH, kd-trees, octrees, bla bla...) Some
% are good for bla, others blank

% Will focus on KD-trees and base the implementation on the proposed
% algorithm in zhou et al.

% Needs to be able to quickly create the data structure aswell for
% photon mapping with dynamic geometry/lighting.

% Needs to exploit the massively parallel architecture of GPGPU's
% which require oh so many thread

% can be done by parallising evaluating the approximate cost of each
% node

\textit{Surface Area Heuristic}, $SAH$, 

% and creation of nodes at lower tree levels

% Still not easily paralisable, requires lots of reduction kernels,
% which do not utilize the GPU fully. Also individual node splitting
% costs can leave 31 threads in a warp waiting for the last one to
% finish. In this thesis persistent threads will be used to solve this
% issue.

