%% • The introduction must be dynamite. 
%% – The reader forms an oppinion of the work right from the start... 
%% • The introduction is an extension of the abstract. 
%% • Should be easy to read and understand 
%% • Should make it easy for anyone to tell
%% – What your paper is about 
%%   – What problem it solves
%%   – Why the problem and solution is interesting and relevant (motivation and context). Is it a long- outstanding problem?
%%   – What is new in your paper and how (much) does it improve on the strongest alternatives/previous work (include a few of the most relevant references here).

%% • Start the introduction with the motivation. Think in large contexts and don’t be afraid to be a poet.
%% • All implications, contributions and keypoints of your work must be included here.
%% • Make it very clear how your work will impact the future of Realistic image generation (will people use it?).
%% • If your work is pioneering, s-p-e-l-l i-t o-u-t.
%% • Briefly make it clear how you evaluate your method in the Results section.
%% • Make sure to explain where your method applies and where it does not apply (limitations).


\chapter{Introduction}

\chapterquote{Focus is a matter of deciding what things you're not
  going to do.}{John Carmack}

% Why global illumination vs local illumination used in most rasterizers

% Why use the GPU? Offloading. The CPU can do a lot of other things
% while the GPU crunch away at those KD trees. Om nom nom nom nom....

% Why even use datastructures on the GPU. They work wonders on the CPU
% but for the GPU they might hinder performance (cite KNN paper and
% dataparallism example with 31 threads waiting)

% In this thesis a brute force ray tracer will first bebuild and
% compared to a basic ray tracer traversing a KD-tree, to blank
% emphasize the importance of datastructures, even on the GPU.

% importance of efficient datastructures, several datastructures have
% been used for raytracing (BVH, kd-trees, octrees, bla bla...) Some
% are good for bla, others blank

% Will focus on KD-trees and base the implementation on the proposed
% algorithm in zhou et al.

% Needs to be able to quickly create the data structure aswell for
% photon mapping with dynamic geometry/lighting.

% Needs to exploit the massively parallel architecture of GPGPU's
% which require oh so many thread

% can be done by parallising evaluating the approximate cost of each
% node

\textit{Surface Area Heuristic}, $SAH$, 

% and creation of nodes at lower tree levels

% Still not easily paralisable, requires lots of reduction kernels,
% which do not utilize the GPU fully. Also individual node splitting
% costs can leave 31 threads in a warp waiting for the last one to
% finish. In this thesis persistent threads will be used to solve this
% issue.

% Raytracing structures on the GPU has been mapped more or less 100%
% from CPU structures. We need to build them for the GPU instead. That
% means rethinking some of the algorithms (like add persistent
% threads.)

% This thesis will explorer different aspects of KD-tree construction,
% how it affects ray tracing performance and how many resources should
% be devoted to KD-tree construction.

% During this I will look at the usefulness of hierachical traversal
% on the GPU, given that a binary data structure is not easy for the
% GPU to handle and breaks coalescence. An exhaustive force ray
% tracer, that performs intersection with all triangles, will be used
% to motivate the use of a hierachical datatsructure, even on the GPU.

% The thesis will look at an alternative to triangle splitting, namely
% triangle division, where the triangle is not split by splitting
% planes, but divided onto each side, by simply adjusting it's
% bounding box.

% Lastly the thesis will try to apply Persistent Threads, as described
% in 'hat', to accelerate SAH computations in the KD-tree construction
% phase.
