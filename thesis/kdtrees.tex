\chapter{KD-Trees}

\chapterquote{In theory, there is no difference between theory and
  practice. But, in practice, there is.}{Jan L. A. van de Snepscheut}

% About this Chapter. Start by motivation. Then explain how kd-trees
% are constructed, including different strategies for choosing the
% spliting plane and doing the actual geometry splitting. Then the
% chapter will end with a discussion of how to implement the
% algorithms efficiently on a SIMT architecture.


% \section{Motivation}

% About KD-trees

% Why KD tree compared to other structures

% Has time and time again been arguibly the best datastructure.

% Vlastimil Havran did an extensive study of available spatial
% subdivision schemes (including regular grids, nested grids, octrees
% and kd-trees) and concludes in his thesis that the kd-tree beats the
% other schemes in most cases. 

% Example of kd-tree flexibility over octree (important in sparse
% scenes / empty space partitioning)

% Cheap intersection / distance to tests. Havran p. 65


\section{Building KD-trees}

% Left balanced non pointer vs pointers

% Choose pointers as that would lead to less memory consumption and it
% places all nodes of the same level in a continues block, making it
% easier to work with them.

% Balanced trees suck, example of partition with high density in one
% side and no triangles in other side. Non balanced tree would leed to
% early exit.

\quotebook{Balancing is optimal only for binary searching, and if all
  nodes have equal access probabilities. Neither of these two
  prerequisites are fulfilled for range queries (such as ray traversal
  and kNN queries), nor for unevenly distributed primitives such as
  photons or triangles.}{wald:04:VVH}

% All the brilliance in KD-tree construction comes down to choosing
% the splitting plane and deciding when to stop.


\subsection{Choosing the Splitting Plane}\label{sec:splittingPlane}

% Infinitely many splitting planes.

% Only interested in those finitely many planes where the resulting
% left and right voxel changes.

% Different splitting planes heuristics.

% Instead of reordering geometry on each split (vertices, normals,
% colors, texcoords, whatnot), we instead simply move an index to the
% geometric primitive. Also decouples the KD-tree creation from the
% geometry attributes and shading.



\subsubsection{Spatial median}

% Split at the spatial median.

% Refered to as the naive implementation in Wald07.

% Easy to understand, easy to implement and nice for getting things
% off the ground.

% Axis to split along can be choosen in a round robin fashion or the
% largest axis can be choosen. (Which initially minimises the surface
% of the children)

% Perform a check to reduce do empty space clipping and facilitate
% early out option for the rays. Requires ``magic splitting
% constants'', which can be very scene specific.


\subsubsection{Surface Area Heuristic}

% SAH assumptions can be seen in Wald07

% Globlly optimal is infeasable for complex scenes and instead a local
% greedy approximation is used.

\begin{displaymath}
  SAH(N \rightarrow \{L, R\}) = C_N + \frac{C_L A_L}{A_N} +
  \frac{C_R A_R}{A_N}
\end{displaymath}

where $C_N$ is the cost of traversing the node itself and is
independent of the splitting plane, $C_L$ is the cost of traversing
the left child nodeand $C_R$ is the cost of traversing the right child
node

% SAH assumes that each split results in 2 leaf nodes, which is
% practically always wrong at high level nodes, therefore
% \citebook{1409079} suggests splitting along the spatial median of the
% nodes longest axis. To do this a preprocess pass is required to
% compute tight AABB's for each triangle.

% SAH yields the best trees.

% SAH calculation optimizations include axis round robin and some damn
% paper I can't remember.

% To avoid having to test the infinitely many splitting planes
% possible, we instead have to choose sensible planes for SAH.

% The sides of the triangles are an obvious choice, however
% precalculating these leads to incorrectly sorting triangles into
% nodes that the triangles do not intersect. (Example of a bounding box
% being inside a node, when the actual triangle is outside)

% A solution is to continuously adjust the bounding box of the split
% goemetry, creating 'perfect split' condidates.

% Another is to simply remove the offending triangles, reafter refered
% to as \textit{false primitives}, from the nodes by a triangle/box
% intersection test during or after creation.

% When moving from upper nodes to lower nodes. Calculate a new tight
% bounding box inside the modified bounding box. Then approximate the
% new surface area by the reduction in BB volume.


\subsubsection{Empty Space Splitting}

% Dynamic empty space threshold, favor early out in the top of the tree.

% huge ray tracing performance improvement in testscene (23% without
% raytracers doing intersection tests at leaf nodes)

% Implementation is 

\subsection{Splitting the Geometry}

This section will look at alternatives to triangle splitting, namely
\textit{triangle division}, where the triangle is not split by
splitting planes, but divided onto each side, by simply adjusting it's
bounding box, and \textit{box inclusion}, where the triangles
themselves are not tested for inclusion, but their bounding boxes are.

% What to do with the triangles caught in the splitting plane.



\subsubsection{Splitting Triangles}

% Normally ppl split.

\subsubsection{Dividing Triangles}

% Novel scheme? Dvide triangles eg. adjust bounding box instead of
% splitting the triangles.

% Spliting will almost always produce 3 triangles, whereas divide will
% always produce only 2.

% Adjusting area heuristic: When we no longer split geometry, SAH
% becomes an even worse approximation. This can be fixed by an
% adjusting area heuristic, where the diagonal of the geometric
% primitives original bounding box is compared to the diagonal of the
% reduced box and the primitives surface area is adjusted accordingly.


\subsubsection{Adjusting Bounding Boxes}

% Simpler than splitting.

% Inspired by the small node step in Zhou

% Naturally increases amount of \textit{false primitives} in the tree, but is
% very cheap.

% Also has an increased change of looping during construction in
% combination with a small max lower size. Fx fairy forest loops using
% adjusting bounding box with a max size of 32 primtives in leaf nodes.

% False primitives can then be removed at a later stage at the cost of
% some extra overhead. Or combine with Divide every n'th step for
% optimal sweetness.

% With the added leaf intersection in the raytracer, extra triangles
% in the leaf nodes become even less important and this method starts
% to shine for dynamic scenes.


% In conclussion:

% Profile the time it takes to create the different trees, especially
% the splitting kernels.

% Compare amount of nodes.

% Tree traversal time.


\subsection{Algorithm for creating KD-tree}

% Argue it can be done in O ( N log N )



\section{Adopting the algorithms for GPGPUs}

% Needs to exploit the dataparallel nature of GPGPU's

% A GPGPU requires (more than) 10^3~10^4 threads for optimal
% performance. To hide latency from texture fetches? cite NVIDIA?

% Use GPU for computations and let CPU handle minor book keeping.

% Handles structures of arrays better than arrays of structures
% chapter 33 \citebook{GPUGEMS2} and coalescence

\subsection{Upper Tree Nodes}

% At upper tree level nodes exploit data parallelism by parallising
% the cost computation over triangles. 

% Use spatiel median splitting for upper nodes. Way faster than SAH
% calculations.

% Creating the KD-tree in BFS will optimize GPU performance at lower
% tree levels, as there would be thousands of nodes created at the
% same time.

% Instead of reducing the sizes of child nodes, I propose a method for
% calculating them directly. This leads to lots of uncoalesced
% lookups, so argue if the GPU is able to properly hide these.

% Building the upper nodes mostly consist of moving data around, and
% not necessarily in a coalesced fashion. This makes it hard to hide
% the latency and will impact performance.

\subsubsection{Empty space splitting}

% Plugable solution, add the new nodes after the ones in nextlist.

% Propagating aabb's downwards

% A good threshold. Did I make it vary and how did that go?

% Perhaps it can be started in it's own stream? At least the actual
% empty space splitting should be able to. Could help out at the early
% tree creation when the GPU is underutilized. Suggest or actually
% try? Would make it an even cheaper optimization.

\subsection{Lower Tree Nodes}

% At lower tree levels where SAH isn't computed parallel, perhaps and
% early out ``good enough'' value/ratio can be given, as done in
% BSP. Might only increase instructions, branching and still wait for
% the slowest thread. (Try and watch it fail) Perhaps without an if
% statement but by arithmetic instead?

% Try persistent threads method!



%% \subsection{Photon Map}

%% \begin{algorithm}
%%   \caption{Photon KD-Tree upper node handling}
%%   \label{alg:PhotonUpperNode}
%%   \begin{algorithmic}
%%     \PROCEDURE{CreateUpperPhotonNodes}
%%               {\VAR{activeList}:list \textit{list of kd-nodes not processed yet}}
%%               {\VAR{lowerList,nextList}:list}{
%%       \STATE{Check size of upper nodes arrays and allocate more if needed.}

%%       \COMMENTIT{Compute bounding box of all active nodes.}
%%       \PARALLELFOR{active node}
%%       \STATE{Compute the bounding box off all it's photons.}
%%       \ENDFOR

%%       \PARALLELFOR{active node}
%%       \STATE{Insert split position, axis and child index}
%%       \ENDFOR

%%       \COMMENTIT{Split photons along axis and store the split index in
%%         the child node}
%%       \FOR{active node}
%%       \STATE{Split the photons in the node \textbf{in parallel}} % Using scan primitives from Sengupta
%%       \ENDFOR

%%       \PARALLELFOR{new node}
%%       \STATE{Split into nextnodes and lowernodes.}
%%       \STATE{Setup links from parents to childrens new position.}
%%       \ENDFOR

%%       \COMMENTIT{CPU bookkeeping}
%%       \COMMENTIT{Perhaps this can be done in a kernel aswell, no need
%%         to do more on the cpu then absolutely necessary.}
%%       \STATE{Copy small child nodes to lowerList}
%%     }
%%   \end{algorithmic}
%% \end{algorithm}


%% \begin{algorithm}
%%   \caption{Lower node handling of the photon kd-tree}
%%   \label{alg:PhotonLowerNode}
%%   \begin{algorithmic}
%%     \PROCEDURE{CreateLowerPhotonNodes}
%%               {\VAR{activeList}:list}
%%               {\VAR{nextList}:list}{
%%                 \PARALLELFOR{node \VAR{i} \textbf{in} activeList}
%%                 \COMMENTIT{Determine optimal splitting plane}
                
%%                 \ASSIGN{\VAR{root}}{\VAR{i.smallRoot}}
%%                 \ASSIGN{\VAR{VVH_0}}{$\|$ \VAR{i.photonBitmap} $\|$}
%%                 \ASSIGN{$VVH$}{$\infty$}
%%                 \STATE{$splitCandidate$}

%%                 \FOR{\textbf{each} splitplane \VAR{s} where s.triangle $\in$ i.photonBitmap}
%%                   \ASSIGN{\VAR{C_L}}{$\| i.triangleSet \cap s.left \|$}
%%                   \ASSIGN{\VAR{C_R}}{$\| i.triangleSet \cap s.right \|$}
%%                   \COMMENTIT{Can't precompute the max and min in splitplane, since triangles change}
%%                   \ASSIGN{$V_L$}{$\sum^k s.max.k - s.min.k + 2 * R$}
%%                   \ASSIGN{$V_L$}{$\sum^k s.max.k - s.min.k + 2 * R$}
%%                   \ASSIGN{$VVH_s$}{$C_L * V_L + C_R * V_R$}
%%                   \IF{$VVH_S < VVH_S$}
%%                     \ASSIGN{$VVH$}{$VVH_S$}
%%                     \ASSIGN{$splitCandidate$}{$s$}
%%                   \ENDIF
%%                 \ENDFOR

%%                 \COMMENTIT{Split to new nodes (Perhaps move this to
%%                   another kernel after doing a scan to determine where
%%                   to place new nodes? Hmm?}

%%                 \ENDFOR
%%               }
%%   \end{algorithmic}
%% \end{algorithm}
