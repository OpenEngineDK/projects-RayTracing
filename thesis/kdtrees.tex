\chapter{KD-Trees}

% About KD-trees

\section{Motivation}

% Why KD tree

% Has time and time again been arguibly the best datastructure.

% Vlastimil Havran did an extensive study of available spatial
% subdivision schemes (including regular grids, nested grids, octrees
% and kd-trees) and concludes in his thesis that the kd-tree beats the
% other schemes in most cases. 

% Example of kd-tree flexibility over octree (important in sparse
% scenes / empty space partitioning)

% Cheap intersection / distance to tests. Havran p. 65


\section{Building KD-trees}

% Left blanced non pointer vs pointers

% Choose pointers as that would lead to less memory consumption and it
% places all nodes of the same level in a continues block, making it
% easier to work with them.

% Balanced trees suck, example of partition with high density in one
% side and no triangles in other side. Non balanced tree would leed to
% early exit.


\quotebook{Balancing is optimal only for binary searching, and if all
  nodes have equal access probabilities. Neither of these two
  prerequisites are fulfilled for range queries (such as ray traversal
  and kNN queries), nor for unevenly distributed primitives such as
  photons or triangles.}{wald:04:VVH}


\subsection{Splitting the geometry}

% Infinitely many splitting planes.

% Only interested in those finitely many planes where the resulting
% left and right voxel changes.


% Unlike \citebook{1409079} triangles will not be split in upper tree
% nodes. Instead two new bounding boxes (maybe triangles) will be
% created and updated to be inside the child nodes bounds. This is
% done to avoid creating extra triangles when splitting (1 triangle
% split often produces 3 triangles), which would result in deeper
% trees. 
% But this method requires us to text that an intersection is in fact
% inside the bb. Do we want that?


% Different splitting planes heuristics.

\subsubsection{Spatial median}

% Split at the spatial median.

% Refered to as the naive implementation in Wald07.

% Easy to understand, easy to implement and nice for getting things
% off the ground.

% Axis to split along can be choosen in a round robin fashion or the
% largest axis can be choosen. (Which initially minimises the surface
% of the children)

% Perform a check to reduce do empty space clipping and facilitate
% early out option for the rays. Requires ``magic splitting
% constants'', which can be very scene specific. (I haven't yet and
% it's not easy!)


\subsubsection{Surface Area Heuristic}

% SAH assumptions can be seen in Wald07

% Globlly optimal is infeasable for complex scenes and instead a local
% greedy approximation is used.

\begin{displaymath}
  SAH(N \rightarrow \{L, R\}) = C_N + \frac{C_L A_L}{A_N} +
  \frac{C_R A_R}{A_N}
\end{displaymath}

where $C_N$ is the cost of traversing the node itself and is
independent of the splitting plane, $C_L$ is the cost of traversing
the left child nodeand $C_R$ is the cost of traversing the right child
node

% SAH assumes that each split results in 2 leaf nodes, which is
% practically always wrong at high level nodes, therefore
% \citebook{1409079} suggests splitting along the spatial median of the
% nodes longest axis. To do this a preprocess pass is required to
% compute tight AABB's for each triangle.

% SAH yields best result


\subsubsection{Voxel Volume Heuristic}

% Voxels have no surface and thus SAH doesn't neceserraly work very
% well with them as primitives. Instead we use VVH

\begin{displaymath}
  \begin{array}{rl}
    VVH(N \rightarrow \{L, R\}) &= C_N + C_L * P_L + C_R * P_R \\
    &= C_N + \frac{C_L * Vol(L \pm r)}{Vol(N \pm r)} + 
    \frac{C_R * Vol(R \pm r)}{Vol(N \pm r)} \\
  \end{array}
\end{displaymath}

hvor $Vol(N \pm r) = \prod_{a \in K} (max(N_a) - min(N_a) + 2 * r)$



\subsection{Algorithm for creating KD-tree}

% O ( N log N )



\section{Nearest neighbour search}

% More on ray tracing in the next section.

% Novel radius scheme? Let the search radius depent on the distance
% the photon has traveled. Nodes that have traveled further represent
% a larger cone from the light, and should therefor cover more
% ground. Check if it really is a novel idea! Now go home! 

% Perhaps use a median of this radius when calculating splits for
% lower nodes? Hope that distribution across neighbouring nodes is
% similar (which it probably is)

% Instead of holding all N neighbours index and distance (squared) in
% memory, can their contribution be calculated while processing and
% thus save lots and lots of shared memory?



\section{Adopting the algorithms for GPGPUs}

% Needs to exploit the dataparallel nature of GPGPU's

% A GPGPU requires (more than) 10^3~10^4 threads for optimal
% performance. To hide latency from texture fetches? cite NVIDIA?

% Use GPU for computations and let CPU handle minor book keeping.

% While the CPU can create it in O ( N log N), some of the operations
% on the gpu, like splitting and sweeping over the dataset, do not
% take O(N) time on the GPU but O(N log N) time. Thus the fastest a
% time complexity possible on a GPU is O ( N log^2 N)

% Handles structures of arrays better than arrays of structures
% chapter 33 \citebook{GPUGEMS2}

% At upper tree level nodes exploit data parallelism by parallising
% the cost computation over triangles or photons. 

% Use spatiel median splitting for upper nodes. Way faster than SAH
% calculations.

% Creating the KD-tree in BFS will optimize GPU performance at lower
% tree levels, as there would be thousands of nodes created at the
% same time.

% At lower tree levels where SAH/VVH isn't computed parallel, perhaps
% and early out ``good enough'' value/ratio can be given, as done in
% BSP. Might only increase instructions, branching and still wait for
% the slowest thread. (Try and watch it fail) Perhaps without an if
% statement but by arithmetic instead?

\subsection{Photon Map}

\begin{algorithm}
  \caption{Photon KD-Tree upper node handling}
  \label{alg:PhotonUpperNode}
  \begin{algorithmic}
    \PROCEDURE{CreateUpperPhotonNodes}
              {\VAR{activeList}:list \textit{list of kd-nodes not processed yet}}
              {\VAR{lowerList,nextList}:list}{
      \STATE{Check size of upper nodes arrays and allocate more if needed.}

      \COMMENTIT{Compute bounding box of all active nodes.}
      \PARALLELFOR{active node}
      \STATE{Compute the bounding box off all it's photons.}
      \ENDFOR

      \PARALLELFOR{active node}
      \STATE{Insert split position, axis and child index}
      \ENDFOR

      \COMMENTIT{Split photons along axis and store the split index in
        the child node}
      \FOR{active node}
      \STATE{Split the photons in the node \textbf{in parallel}} % Using scan primitives from Sengupta
      \ENDFOR

      \PARALLELFOR{new node}
      \STATE{Split into nextnodes and lowernodes.}
      \STATE{Setup links from parents to childrens new position.}
      \ENDFOR

      \COMMENTIT{CPU bookkeeping}
      \COMMENTIT{Perhaps this can be done in a kernel aswell, no need
        to do more on the cpu then absolutely necessary.}
      \STATE{Copy small child nodes to lowerList}
    }
  \end{algorithmic}
\end{algorithm}


\begin{algorithm}
  \caption{Lower node handling of the photon kd-tree}
  \label{alg:PhotonLowerNode}
  \begin{algorithmic}
    \PROCEDURE{CreateLowerPhotonNodes}
              {\VAR{activeList}:list}
              {\VAR{nextList}:list}{
                \PARALLELFOR{node \VAR{i} \textbf{in} activeList}
                \COMMENTIT{Determine optimal splitting plane}
                
                \ASSIGN{\VAR{root}}{\VAR{i.smallRoot}}
                \ASSIGN{\VAR{VVH_0}}{$\|$ \VAR{i.photonBitmap} $\|$}
                \ASSIGN{$VVH$}{$\infty$}
                \STATE{$splitCandidate$}

                \FOR{\textbf{each} splitplane \VAR{s} where s.triangle $\in$ i.photonBitmap}
                  \ASSIGN{\VAR{C_L}}{$\| i.triangleSet \cap s.left \|$}
                  \ASSIGN{\VAR{C_R}}{$\| i.triangleSet \cap s.right \|$}
                  \COMMENTIT{Can't precompute the max and min in splitplane, since triangles change}
                  \ASSIGN{$V_L$}{$\sum^k s.max.k - s.min.k + 2 * R$}
                  \ASSIGN{$V_L$}{$\sum^k s.max.k - s.min.k + 2 * R$}
                  \ASSIGN{$VVH_s$}{$C_L * V_L + C_R * V_R$}
                  \IF{$VVH_S < VVH_S$}
                    \ASSIGN{$VVH$}{$VVH_S$}
                    \ASSIGN{$splitCandidate$}{$s$}
                  \ENDIF
                \ENDFOR

                \COMMENTIT{Split to new nodes (Perhaps move this to
                  another kernel after doing a scan to determine where
                  to place new nodes? Hmm?}

                \ENDFOR
              }
  \end{algorithmic}
\end{algorithm}

\subsection{Triangle Map}

\begin{itemize}
\item
Indel i segmenter af smallnode max size * 4. Fortæl kernels, der
arbejder på segmenter, hvor mange tråde de max kan forvente. Så kan
compileren optimere local memory.

segments [2, 1, 3, 2]
addrs    [0, 2, 3, 6, 8]
owners   [0, 0, 1, 1, 0, 0, 1, 0] -> [0, 0, 1, 2, 2, 2, 3, 3]

Beregn trekants info:

Først trekants index for segment i med nodeID = owners[i]:

offset = SEGMENT_SIZE * (i - addrs[nodeID])
index[i] = photonInfo[nodeID].x + offset
range[i] = min(SEGMENT_SIZE, photonInfo[nodeID].y - offset);


\item
Reduce the nodes bounding boxes, first pr segment, then do a segmented
reduction. (Possibly unroll some of the loops)

\item
Sorter trekanter til deres nye placeringer. SplitSide er et array af
størrelse activeTris * 2 som i node chunks indeholder hvor mange
trekanter en trekant bliver splittet i på højre og venstre
side. Prefix sum giver så positionen for disse trekanter.

Eks

Node 1 og 2

Owner     [1, 1, 1, 1, 2, 2, 2, 2] // Den knude der ejer trekanten.\\
side      [l, l, r, r, l, l, r, r] // Hvilken side splittet er på.\\
trekant   [0, 1, 0, 1, 2, 3, 2, 3] // Trekanten der kigges på.\\
splits    [2, 1, 1, 2, 1, 0, 0, 1] // Hvor mange trekanter på siden splittet resulterer i.\\
addr      [0, 2, 3, 4, 6, 7, 7, 7, 8] // Start index for de enkelte trekanter og samlede antal trekanter. Beregnet ved prefix sum\\

Så vi har 8 trekanter efter splitsne.

Muligvis lav et rent left til right split. Burde give bedre coalescence.

\item
Sorter nye child node trekanter ud i et result array og tilpas addr arrayet

Størrelsen på knuden kan ses ud fra addr[index + range] - addr[index]
for childnodes. Med en small size på 1 vil det ovenstående eksempel give

Node3.child = 3
Node4.child = 3
Node5.child = 1
Node6.child = 1

leaf:      [0, 0, 0, 0, 1, 1, 1, 1] // Er trekanten del af et leaf
leafSplit: [0, 0, 0, 0, 1, 0, 0, 1] // Splits
leafAddr:  [0, 0, 0, 0, 0, 1, 1, 1, 2] // Leaf list addr
addr:      [0, 2, 3, 4, 6, 6, 6, 6, 6] // Nye addrs. addr -= leafAddr.

Så der bliver 6 nye knuder og 2 leaf knuder.

``Færdige'' leaf trekanter bliver nødt til at blive kopieret til et
andet array, fordi der hele tiden bliver produceret nye trekanter
(Damn splits)

\item
Opret child nodes or sortere dem efter leaf/nonLeaf.

\item
Bøvle med leaf index/range.

\end{itemize}

