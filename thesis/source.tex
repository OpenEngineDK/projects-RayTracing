\chapter{Obtaining the Project}

A short demo of the project can be found on
\url{http://www.youtube.com/watch?v=h9gecC5woxs} or the root of the accompanying
CD.

The source code for this master's thesis is also located on the accompanying CD
in the folder RayTracingDynamicScenes. The project has been developed in the
open source 3D engine OpenEngine and I have therefore not implemented everything
found in RayTracingDynamicScenes. The implementation of my ray tracer can be
found under the project/RayTracing folder, which contains the scene setup, and
under extensions/PhotonMapping, where the code for construction kd-trees and ray
tracing them is located. The project has been compiled and tested on Max OS X
and Ubuntu.

If the CD is not present, then the project can be obtained by installing
OpenEngine and then installing the project. OpenEngine requires Python, CMake,
Boost and the version control system Darcs to be installed on the system. My
project additionally requires that SDL, CUDA 3.1 and FreeImage be installed on
the system.

To install OpenEngine simply run the following command in a terminal

\terminalCommand{darcs get http://openengine.dk/code/openengine}

This will place the OpenEngine source in a folder named \textit{openengine} in
the current working directory. cd into that directory and execute the commands

\terminalCommand{chmod +x dist.py}

\terminalCommand{chmod +x make.py}

The project and its dependencies can then be installed with the command

\terminalCommand{./dist.py install proj:RayTracing}

The project can then be compiled by running 

\terminalCommand{./make.py}

and run with the command

\terminalCommand{./build/RayTracing/RayTracing}
