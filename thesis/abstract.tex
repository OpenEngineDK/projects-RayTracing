%% Isbrief:200-300words. 
%% • Summarizes your work:
%% – What problem does this paper solve?
%% – Why is this problem important for medical image computing (the context and motivation)? 
%% – How does your method work? 
%% – How does it differ from previous work? 
%% – How much better than previous work is it?
%% • There are usually no references in the abstract
%% • The abstract should be readable and understandable by non-experts


\section*{Abstract}

This thesis concerns itself with raytracing and doing that efficiently
on wide SIMT\footnote{single instruction, multiple threads}
hardware. It is motivated by the ever increasing interest in
raytracing and global illumination for creating effects in movies, but
also the increased usage of 2D and 3D ray tracing used in modern
computer games.

The thesis present a survey of different techniques for creating
hierarchical ray tracing acceleration structures, specifically
KD-trees, and how to do this efficiently on the wide SIMT hardware. A
ray tracer will be used to test the quality of the differently created
KD-trees. To give a fair comparison the time spend ray tracing
and time spend creating the tree in a dynamic scene, both
generel and SIMT specific optimizations have been applied to the ray
tracer and acceleration structure.

While previous work in the area has focused on creating optimal
KD-trees, the goal of this thesis is to explorer compromises made
during tree construction in order to efficiently ray trace dynamic
scenes.

% Show that to achive raytracing of dynamic scenes, focus should not
% be on creating 'optimal' trees, but instead on creating good trees
% fast.






% Efficiently create KD trees on the GPU and ray trace them.

% KDtree focus will be on optimizations and the tradeoff between
% spending extra time on the kd-tree instead of ray tracing.


