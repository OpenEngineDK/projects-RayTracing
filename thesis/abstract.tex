%% Isbrief:200-300words. 
%% • Summarizes your work:
%% – What problem does this paper solve?
%% – Why is this problem important for medical image computing (the context and motivation)? 
%% – How does your method work? 
%% – How does it differ from previous work? 
%% – How much better than previous work is it?
%% • There are usually no references in the abstract
%% • The abstract should be readable and understandable by non-experts


\begin{center}
\begin{minipage}{0.7\textwidth}
\section*{Abstract}
This thesis is aimed at raytracing dynamic scenes and doing that
efficiently while harnesing the massive power of todays modern
graphics cards. It is motivated by the ever increasing interest in
raytracing and global illumination for creating effects in movies, but
also the increased usage of 2D and 3D ray tracing in modern computer
games.

The thesis explores different techniques for creating hierarchical ray
tracing acceleration structures, specifically the binary kd-trees, and
how to do this efficiently on wide SIMT hardware.

While previous work in the area has focused on creating optimal
acceleration structures, the goal of this thesis is to explorer the
tradeoffs between spending time producing structures of high quality
or more quickly produce a lower quality acceleration structure.

A ray tracer will be used to compare the quality of the different
kd-trees.

\fixme{Just write optimized ray tracer and place the following in
  intro?}  

To give a fair comparison of the time spend ray tracing and time spend
creating the tree, both general and SIMT specific optimizations have
been applied to the ray tracer and kd-tree.

\end{minipage}
\end{center}

% Show that to achive raytracing of dynamic scenes, focus should not
% be on creating 'optimal' trees, but instead on creating good trees
% fast.






% Efficiently create KD trees on the GPU and ray trace them.

% KDtree focus will be on optimizations and the tradeoff between
% spending extra time on the kd-tree instead of ray tracing.


