%% Isbrief:200-300words. 
%% • Summarizes your work:
%% – What problem does this paper solve?
%% – Why is this problem important for medical image computing (the context and motivation)? 
%% – How does your method work? 
%% – How does it differ from previous work? 
%% – How much better than previous work is it?
%% • There are usually no references in the abstract
%% • The abstract should be readable and understandable by non-experts


\begin{center}
\begin{minipage}{0.7\textwidth}
\section*{Abstract}

This thesis concerns itself with raytracing and doing that efficiently
while exploiting the massive power of todays modern graphics cards. It
is motivated by the ever increasing interest in raytracing and global
illumination for creating effects in movies, but also the increased
usage of 2D and 3D ray tracing used in modern computer games.

The thesis present a survey of different techniques for creating
hierarchical ray tracing acceleration structures, specifically the
binary kd-trees, and how to do this efficiently on the wide SIMT
hardware. A ray tracer will be used to test the quality of the
differently created kd-trees.

While previous work in the area has focused on creating optimal
acceleration structures, the goal of this thesis is to explorer if
spending less time on optimizing the acceleration structures for ray
tracing, can result in an overall performance increase in a dynamic
scene.

To give a fair comparison of the time spend ray tracing and time spend
creating the tree, both generel and SIMT specific optimizations have
been applied to the ray tracer and kd-tree.

\end{minipage}
\end{center}

% Show that to achive raytracing of dynamic scenes, focus should not
% be on creating 'optimal' trees, but instead on creating good trees
% fast.






% Efficiently create KD trees on the GPU and ray trace them.

% KDtree focus will be on optimizations and the tradeoff between
% spending extra time on the kd-tree instead of ray tracing.


