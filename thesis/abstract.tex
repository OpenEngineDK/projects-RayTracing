%% Isbrief:200-300words. 
%% • Summarizes your work:
%% – What problem does this paper solve?
%% – Why is this problem important for medical image computing (the context and motivation)? 
%% – How does your method work? 
%% – How does it differ from previous work? 
%% – How much better than previous work is it?
%% • There are usually no references in the abstract
%% • The abstract should be readable and understandable by non-experts


\begin{center}
\begin{minipage}{0.7\textwidth}
\section*{Abstract}

This thesis is aimed at ray tracing dynamic scenes and doing that efficiently
while harnesing the massive power of today's graphics cards. It is motivated by
the ever increasing interest in ray tracing and global illumination for creating
effects in movies, but also the increased usage of 2D and 3D ray tracing in
modern computer games.

This thesis explores different techniques for creating hierarchical ray tracing
acceleration structures, specifically the binary kd-trees, and how to create
these efficiently on graphics hardware.

While previous work with kd-trees has focused on creating high quality
acceleration structures, the goal of this thesis is to explore the relationship
between construction speed and tree quality. I will focus on dynamic scenes,
where the kd-tree must be rebuild before each image rendered, and investigate if
the overall time to ray trace a scene can be increased by producing acceleration
structures much faster, but at a lower quality.

The quality of a kd-tree will be defined by how fast it can be used to ray trace
a scene. An optimized ray tracer will therefore be developed as part of the
thesis to evaluate the quality of the kd-trees produced. The total time spent
construction and ray tracing the kd-tree will be used to estimate if there is a
potential performance gain to be had by quickly producing kd-trees of lower
quality.

\end{minipage}
\end{center}


