%% Isbrief:200-300words. 
%% • Summarizes your work:
%% – What problem does this paper solve?
%% – Why is this problem important for medical image computing (the context and motivation)? 
%% – How does your method work? 
%% – How does it differ from previous work? 
%% – How much better than previous work is it?
%% • There are usually no references in the abstract
%% • The abstract should be readable and understandable by non-experts


\begin{center}
\begin{minipage}{0.7\textwidth}
\section*{Abstract}

This thesis is aimed at raytracing dynamic scenes and doing that efficiently
while harnesing the massive power of todays graphics cards. It is motivated by
the ever increasing interest in raytracing and global illumination for creating
effects in movies, but also the increased usage of 2D and 3D ray tracing in
modern computer games.

The thesis explores different techniques for creating hierarchical ray tracing
acceleration structures, specifically the binary kd-trees, and how to create
these efficiently on graphics hardware.

\fixme{Quality is defined as ray tracing speed}

While previous work with kd-trees has focused on creating optimal acceleration
structures, the goal of this thesis is to explorer the tradeoffs between
spending time producing structures of high quality or more quickly produce a
lower quality acceleration structure. This tradeoff can be very important in
dynamic scenes, where the acceleration structure may have to be rebuild for
every frame.

As part of the thesis an optimized ray tracer will be developed to evaluate the
quality of the kd-trees produced and to estimate if there is a potential
performance gain to be had by quickly producing kd-trees of lower quality.

\end{minipage}
\end{center}

% Show that to achive raytracing of dynamic scenes, focus should not
% be on creating 'optimal' trees, but instead on creating good trees
% fast.






% Efficiently create KD trees on the GPU and ray trace them.

% KDtree focus will be on optimizations and the tradeoff between
% spending extra time on the kd-tree instead of ray tracing.


