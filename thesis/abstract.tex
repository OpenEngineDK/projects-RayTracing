%% Isbrief:200-300words. 
%% • Summarizes your work:
%% – What problem does this paper solve?
%% – Why is this problem important for medical image computing (the context and motivation)? 
%% – How does your method work? 
%% – How does it differ from previous work? 
%% – How much better than previous work is it?
%% • There are usually no references in the abstract
%% • The abstract should be readable and understandable by non-experts

\begin{center}
\begin{minipage}{0.7\textwidth}
\vspace{6cm}

\section*{Abstract}

The topic of this thesis is ray tracing dynamic scenes and doing that
efficiently while harnessing the massive computational power of today's graphics
cards. It is motivated by the ever increasing interest in ray tracing and global
illumination for creating effects in movies, but also the increased usage of 2D
and 3D ray tracing in modern computer games.

The thesis explores different techniques for creating hierarchical acceleration
structures for ray tracing, specifically binary kd-trees, and how to create
these efficiently on graphics processing units.

The quality of a kd-tree is defined as the speed with which it can be used to
ray trace a scene and a lot of previous research has been focused on creating
kd-trees of high quality, which usually results in an increased construction
time. The goal of this thesis is to explore the relationship between
construction speed and tree quality for kd-trees. I will focus on dynamic
scenes, where the kd-tree must be rebuild before an image is rendered, and
investigate if the overall time to ray trace a scene can be increased by
producing acceleration structures faster, but at a lower quality.

As part of the thesis I will develop an optimized ray tracer to evaluate the
quality of the kd-trees produced. The total time spent constructing and ray
tracing the kd-tree will be used to estimate if there is a potential performance
gain to be had by quickly producing kd-trees of lower quality.

\end{minipage}
\end{center}


