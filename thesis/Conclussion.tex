%% Briefly motivate your work and its context. 

%% Summarize your work/method and the results (it was X times better than the
%% strongest alternative).

%% Spell out the implications for and the impacts on ”ray tracing”. Why it is
%% useful and interesting?

%% Keep in mind that the conclusion will be the last the reviewer reads: You
%% want to leave him with a good impression. - Don’t end the conclusion by
%% saying ”My method can’t do that...” but end by stating its importance and
%% impact.


\chapter{Conclusion}

\chapterquote{The best thing about a boolean is even if you are wrong,
  you are only off by a bit.}{Anonymous}

%\fixme{Conclusion must mirror the goals in chp 1! Do it!}

In this thesis I have presented several different algorithms for constructing
kd-trees and compared their construction speed and their quality, using an
optimized hierarchical ray tracer and applying the kd-tree constructors to
dynamic scenes.

% Conclude on ray tracers

% Exhaustive only usefull in small scenes

In \refchapter{chp:results} I tested an exhaustive ray tracer and compared it
against hierarchical ray tracer solutions. The result was that the exhaustive
ray tracer did not scale nearly as well as hierarchical approaches and I can
therefore conclude that acceleration structures should still be used for dynamic
scenes.

% Short stack overhead making it only useful in large scenes.

Both packets and leaf skipping proved to be reliable optimizations that only
provided a small overhead in the very simple Cornell Box scene, but otherwise
always sped up ray tracing. The short-stack optimization did not provide such a
stable increase in speed. Even for the fairly complex reflecting dragon scene,
the overhead from maintaining a stack was still too much to provide any
performance boost. However in the Sponza scene, where the tree was twice as
large compared to the dragon scene, adding a short-stack allowed the rays to
avoid traversing large parts of the tree and thus improve their overall
performance. We can thus conclude that while the short-stack optimization may
not always provide a decreased rendering time, using it allows the ray tracer to
perform better in large scenes.


% Conclude on kd-trees

I have during this thesis presented 4 different algorithms for splitting a node
in a kd-tree and 2 schemes for associating triangles with the resulting child
nodes.

%% Conclude overall! evaluate the tradeoff between construction speed and tree
%% quality.

QED




%% This section is usually combined with the conclusion section described next.

%% State immediate extensions of your work, things you did not have time to do,
%% or related well-known problems.

%% If you state a future work problem that you don’t want others to steal, say
%% you are already working on it or that it is ”almost” in submission elsewhere.


\chapter{Future Work}\label{chp:future}

\chapterquote{Software is like entropy: It is difficult to grasp,
  weighs nothing, and obeys the Second Law of Thermodynamics; i.e., it
  always increases.}{Norman Augustine}

New node layout

Optimize triangle divide by excluding uninteresting test cases.

Optimize SAH/SSAH





%% KD/BVH combination trees: still only carry information for one dimension, but
%% also provide near and far planes, useful for estimating the advancement of
%% the ray's $t_{min}$.
