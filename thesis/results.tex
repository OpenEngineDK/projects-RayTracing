% Tabels of nodes and triangled visited. Perhaps some statistics over
% wasted iterations by threads waiting for other threads in the warp
% to finish. The high triangle count pr leaf optimization may be
% related to this and makes using highly optimized splitting plane
% calculations in the lower nodes useless.

\chapter{Results}\label{chp:results}

\chapterquote{It’s hardware that makes a machine fast. It’s software
  that makes a fast machine slow.}{Craig Bruce}

% Intro

In this chapter I will compare the construction speed and the quality of the
different kd-tree construction schemes implemented as part of this thesis.

% Hardware

The implementations have been tested on a dual core Intel Core i7 2.66GHz CPU
with 4GB RAM. The GPU used is an NVIDIA 330M with CUDA Compute Capability 1.2, 6
multiprocessors running at 1.1GHz and 512MB RAM. CUDA 3.1 was installed on the
machine. All tests where performed with a screen resolution of 640x480, which
equals 307200 primary rays traced into the scene.

\begin{figure}
  \centering
  \subfloat[The Cornell Box - 36 Triangles.]{
    \includegraphics[width=0.3\textwidth]{cornellBox}
  }
  \subfloat[The Reflecting Stanford Dragon - 203k Triangles.]{
    \includegraphics[width=0.3\textwidth]{semiReflectingDragon}
  }
  \subfloat[Sponza - 279k Triangles.]{
    \includegraphics[width=0.3\textwidth]{sponza}
  }
  \caption[Test scenes.]{The 3 test scenes used in this chapter.}
  \label{fig:testScenes}
\end{figure}

% Test scenes: Cornell, reflecting dragon2 and sponza

The tests will be conducted on three different test scenes shown in
\reffig{fig:testScenes}. \textit{The Cornell Box} scene was chosen for its
simplicity. With only 36 triangles the scene should be very fast to render and
allows me to compare the performance of the ray tracer implementations on a
simple scene. The \textit{Sponza} scene was chosen for its complexity. At 279k
triangles it will test both the quality and the construction speed of the
kd-tree constructors in large scenes. The last scene tested is \textit{The
  Reflecting Stanford Dragon} consisting of 203k triangles. The reflecting
geometry in the scene spawns 189572 reflection rays and is used to test
the importance of a high quality kd-tree when more rays than just the primary
are traced.

% Rest of this chapter

In \refsection{sec:evaluateRayTracer} I will evaluate the different ray tracer
implementations and the impact of the optimizations described in
\refsection{sec:hierarchicalTraversal}. I have chosen to implement the
short-stack optimization as a seperat ray tracer, instead of applying it to the
kd-restart ray tracer. This is merely a design decision and has no impact on the
actual results presented below. 
%, except for presenting them more clearly, but how is it clear?
The overall fastest ray tracer will be used in the following sections to
evaluate the quality of the kd-trees constructed. In
\refsection{sec:evaluateUpperTree} I will compare quality and construction speed
of the different construction schemes for the upper part of the kd-tree. The
upper tree configuration that performs best, will be used in
\refsection{sec:evaluateUpperTree}, where the kd-tree's lower parts will be
constructed using SAH, SSAH and no construction scheme.

\fixme{Dynamic scenes, so best kd-tree is the one with lowest summed
  construction + ray tracing time.}

\section{Evaluate Ray Tracers}\label{sec:evaluateRayTracer}

\Reffig{fig:rayTracerEvaluation} shows the time it took different ray tracer
configurations to render each of the three test scenes. The kd-tree
configuration used to produce kd-trees for the hierarchical ray tracers is
described in the figure's caption.

\newcommand{\tabelMoeller}{
  \begin{tabular}{c}
    Moeller- \\ Trumbore
  \end{tabular}
}

\begin{figure}
  \centering
  \begin{minipage}{\textwidth}
    \centering
    \SetTabelTextSize
    \begin{tabular} {r | c | c | c || c || c || c ||}
      \tabelParam{c}{\textit{Ray} \\ \textit{Tracer:}} &
      \tabelParam{c}{\textit{Ray/triangle}\\\textit{intersection:}} &
      \tabelParam{c}{\textit{Packets:}} &
      \tabelParam{c||}{\textit{Leaf} \\ \textit{skipping:}} &
      \tabelScene{Cornell \\ Box \\ 1.0ms (3)} & 
      \tabelScene{Reflecting \\ Dragon \\ 150ms (49909)} & 
      \tabelScene{Sponza \\ 263ms (98779)}\\
      \hline
      \multirow{4}{*}{Exhaustive\footnote{The exhaustive ray tracer only ray traces the first 4096 triangles in a scene. Even with this hard upper limit it is clear that an exhaustive approach will not work for detailed scenes.}} & 
        \multirow{2}{*}{\tabelMoeller} & No & N/A & 11.2ms & \worstResult{1118ms} & \worstResult{964ms} \\
      \cline{3-7}
      & & Yes & N/A & \worstResult{11.6ms} & 1114ms & \worstResult{964ms} \\
      \cline{2-7}
      & \multirow{2}{*}{Woop} & No & N/A & \bestResult{9.3ms} & \bestResult{898ms} & \bestResult{650ms}\\
      \cline{3-7}
      & & Yes & N/A & 9.7ms & 903ms & 651ms\\
      \hline
      \hline

      % KD-Restart
      \multirow{8}{*}{KD-Restart} & \multirow{4}{*}{\tabelMoeller} & \multirow{2}{*}{No} & No & 17.9ms & \worstResult{790ms} & \worstResult{592ms} \\
      \cline{4-7}
      & & & Yes & \worstResult{18.1ms} & 486ms & 287ms \\
      \cline{3-7}
      & & \multirow{2}{*}{Yes} & No & 17.2ms & 579ms & 424ms \\
      \cline{4-7}
      & & & Yes & 17.4ms & \bestResult{355ms} & \bestResult{196ms} \\
      \cline{2-7}
      & \multirow{4}{*}{Woop} & \multirow{2}{*}{No} & No & 15.5ms & 690ms & 549ms \\
      \cline{4-7}
      & & & Yes & 15.8ms & 500ms & 287ms \\
      \cline{3-7}
      & & \multirow{2}{*}{Yes} & No & \bestResult{14.6ms} & 495ms & 342ms \\
      \cline{4-7}
      & & & Yes & 14.8ms & \bestResult{355ms} & \bestResult{198ms} \\
      \hline
      \hline
      
      % Short-Stack
      \multirow{8}{*}{Short-Stack} & \multirow{4}{*}{\tabelMoeller} & \multirow{2}{*}{No} & No & 20.0ms & \worstResult{808ms} & \worstResult{595ms} \\
      \cline{4-7}
      & & & Yes & \worstResult{20.2ms} & 503ms & 266ms \\
      \cline{3-7}
      & & \multirow{2}{*}{Yes} & No & 19.5ms & 590ms & 421ms \\
      \cline{4-7}
      & & & Yes & 19.8ms & \bestResult{360ms} & \bestResult{167ms} \\
      \cline{2-7}
      & \multirow{4}{*}{Woop} & \multirow{2}{*}{No} & No & 18.5ms & 738ms & 560ms \\
      \cline{4-7}
      & & & Yes & 18.7ms & 522ms & 270ms \\
      \cline{3-7}
      & & \multirow{2}{*}{Yes} & No & \bestResult{17.8ms} & 523ms & 348ms \\
      \cline{4-7}
      & & & Yes & 18.1ms & 368ms & 171ms \\
      \hline
    \end{tabular}\par
    \vspace{-0.75\skip\footins}
    \renewcommand{\footnoterule}{}
  \end{minipage}
  \caption[Ray tracing results.]{The table shows the time it took different ray
    tracer configurations to render each of the three test scenes.  The bold
    faced render times performed the best for a given ray tracer and scene. The
    ones in red performed the worst. The kd-trees for the scenes were
    constructed with empty space maximization enabled and a threshold of 25\%,
    the triangle/node association scheme employed was dividing. The threshold for
    the lower tree was 32, but no lower tree was constructed. The size of the
    kd-tree is reported in parenthesis below the name of the scene. The size of
    the short stack was 4 elements, which has been found to perform well in
    practice.}
  \label{fig:rayTracerEvaluation}
\end{figure}

% Evaluate the ray tracers for the individual scenes

\Reffig{fig:rayTracerEvaluation} shows that The exhaustive ray tracer generelly
performs worst of all the ray tracers. This was to be expected, since the
exhaustive ray tracers time complexity is $O(nm)$ and the hierarchical ray
tracers' are $O(n \log m)$, for $n$ rays and $m$ triangles. As expected Woop's
simpler Triangle/Ray intersection performs better than Moeller-Trumbore's in all
cases. The screen space packet optimization has also been applied to the
exhaustive ray tracer, but generelly causes a small overhead instead of a
performance improvement. This was to be expected, since all rays intersect the
triangles in the same order, and therefore it does not matter which rays are
traced together. In the Reflecting Dragon scene we see a small speed increase
when using packets. This may be the result of fewer warps tracing reflection
rays, but can also be caused by an acceptable deviation in timing precision.

In the Cornell Box scene, the kd-restart implementation performed worse than the
exhaustive ray tracer, even without taking the kd-tree construction time into
account. This is not entirely unexpected, since the exhaustive ray tracer only
needs to intersect every ray with 36 triangles, where the kd-restart
implementation needs to first traverse the root node, before it can begin
ray/triangle intersection in leafs that can reference up to 32 triangles. In the
more complex scenes, kd-restart outperformed exhaustive ray tracing by a wide
margin. Unsurprisingly the configuration consisting of Moeller-Trumbore
intersection, no packets and no leaf skipping performed the worst. It was a bit
more surprising to see that with packets and leaf skipping enabled, it did not
matter much if Moeller-Trumbore's or Woop's intersection method was
used. However, this makes sense since the rays in the same warp are spatially
close and are therefore able to skip most of the leaf nodes they visit, and thus
a lot of intersection computations, before reaching their intersection point.

The short-stack implementation performed very similar to kd-restart. In the
simple Cornell Box scene it performed slightly worse, due to the overhead of
maintaining a stack. In the Reflecting Dragon scene, it performed roughly
similar to kd-restart and in the Sponza scene it outperformed both the
exhaustive ray tracer, kd-restart and was about 70\% faster than the an
unoptimized kd-restart implementation.

Overall the short-stack implementation performed the best. While it may have
been outperformed in the Cornell Box, it was by far the best ray tracer in the
complex Sponza scene. I will therefore use the short-stack implementation, with
Moeller-Trumber intersection, packets and leaf skipping enabled, to evaluate the
quality of kd-trees in the following sections.


\section{Evaluate Upper Tree Creation}\label{sec:evaluateUpperTree}

% Upper tree creation parameters

% Not creating lower nodes (using None)

\begin{figure}
  \centering
  \SetTabelTextSize
  \begin{tabular} {c | c | c || c || c ||}
    % Titel bar: Association scheme | empty space maximization | threshold |
    % Cornell | dragon | sponza
    \multicolumn{1}{c}{\begin{tabular}{c}\textit{Association} \\ \textit{scheme:}\end{tabular}} &
    \multicolumn{1}{c}{\begin{tabular}{c}\textit{Empty Space} \\ \textit{Maximization:}\end{tabular}} &
    \begin{tabular}{c}\textit{Empty}\\\textit{Space} \\ \textit{Threshold:}\end{tabular} &
    \tabelScene{Reflecting \\ Dragon} &
    \tabelScene{Sponza}\\
    \hline % Titel end
    \multirow{4}{*}{Dividing} & No & N/A & \worstResult{146/385ms (38k)} & \worstResult{258/220ms (87k)}\\
    \cline{2-5}
    & \multirow{3}{*}{Yes} & 15\% & 150/376ms (56k) & 263/167ms (104k) \\
    \cline{3-5}
    & & 25\% & 150/355ms (50k) & 263/167ms (99k)\\
    \cline{3-5}
    & & 35\% & 150/357ms (45k) & 263/167ms (95k)\\
    \hline
    \multirow{4}{*}{Box Inclusion} & No & N/A & 116/397ms (41k) & 213/241ms (112k) \\
    \cline{2-5}
    & \multirow{3}{*}{Yes} & 15\% & 121/386ms (59k) & 219/182ms (130k) \\
    \cline{3-5}
    & & 25\% & 121/373ms (53k) & 219/182ms (125k)\\
    \cline{3-5}
    & & 35\% & \bestResult{121/366ms (48k)} & \bestResult{219/180ms (120k)}\\
    \hline
  \end{tabular}
  \caption[Upper tree creation results.]{The entries in the table shows the time
    it took to create the kd-tree in miliseconds, the time it took to render the
    scene and the number of nodes in the kd-tree is the number in
    parenthesis. The scene was ray traced using the short stack configuration
    that performed best above. To keep the lower tree construction phase from
    having as little influence on teh results as possible, no lower tree will be
    constructed and lower tree threshold has been set to 32.}
  \label{fig:upperResults}
\end{figure}


\section{Evaluate Lower Tree Creation}\label{sec:evaluateLowerTree}

% Lower node creation

\begin{figure}
  \centering
  \SetTabelTextSize
  \begin{tabular}{c |c | c || c || c ||}
    \tabelParam{c}{\textit{Bit Mask:}} &
    \tabelParam{c}{\textit{Splitting} \\ \textit{Scheme:}} &
    \tabelParam{c||}{$C_{N}:$} &
    \tabelScene{Reflecting \\ Dragon} &
    \tabelScene{Sponza}\\
    \hline % Titel end
    \multirow{7}{*}{32bit} & None & N/A & \bestResult{121/357ms (48.1k)} & \bestResult{219/179ms (120k)}\\
    \cline{2-5}
    & \multirow{3}{*}{SAH} & 24 & 273/358ms (48.2k) & 616/180ms (121k)\\
    \cline{3-5}
    & & 16 & 273/359ms (48.3k) & 648/180ms (129k)\\
    \cline{3-5}
    & & 8 & \worstResult{280/358ms (50k)} & \worstResult{775/181ms (167k)}\\
    \cline{2-5}
    & \multirow{3}{*}{SSAH} & 24 & 159/382ms (126k) & 305/201ms (287k)\\
    \cline{3-5}
    & & 16 & 162/393ms (143k) & 309/207ms (313k)\\
    \cline{3-5}
    & & 8 & 165/401ms (168k) & 318/227ms (394k)\\
    \hline \hline
    \multirow{7}{*}{64bit} & None & N/A & \bestResult{98/464ms (19k)} & \bestResult{160/193ms (38k)}\\
    \cline{2-5}
    & \multirow{3}{*}{SAH} & 40 & 1113/462ms (20k) & 2393/194ms (41k)\\
    \cline{3-5}
    & & 32 & 1113/463ms (20) & 2530/193ms (43k)\\
    \cline{3-5}
    & & 24 & \worstResult{1126/467ms (20)} & \worstResult{2816/190ms (48k)}\\
    \cline{2-5}
    & \multirow{3}{*}{SSAH} & 40 & 326/416ms (92k) & 609/200ms (162k)\\
    \cline{3-5}
    & & 32 & 328/423ms (102k) & 613/204ms (177k) \\
    \cline{3-5}
    & & 24 & 331/426ms (110k) & 616/206ms (191k)\\
    \hline
  \end{tabular}
  \caption[Lower tree creation results.]{Lower tree creation results. --What is
    seen and what parameters where used--}
  \label{fig:lowerResults}
\end{figure}

% Splitting Schemes:

% Profile the time it takes to create the different trees, especially
% the splitting kernels.

% Compare amount of nodes.

% Tree traversal time.

% Warp coherence: balance between traversal steps and intersections =
% high amount of nodes in the leafs ie even less important to spend
% alot of time on that last split.


